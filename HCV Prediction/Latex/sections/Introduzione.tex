\section{Introduzione}\label{sec:Introduction}
L'obbiettivo di questa introduzione è quella di fornire un quadro generale del problema dell'HCV e della descrizione dei dati raccolti.
\subsection{Contestualizzazione}\label{ssec:Context}

L'epatite C è un fenomeno altamente presente in Egitto poichè vi è stata una campagna di massa per la diagnosi di schistosomiasi in cui sono state utilizzate delle siringhe impropriamente sterilizzate. Questo tipo di malattia colpisce il fegato ed è estremamente difficile da rilevare poichè il più delle volte è asintomatico nei primi decenni. Tuttavia, i pazienti i cui dati sono stati riportati nel dataset, hanno una qualche forma di fibrosi e sono stati sottoposti ad un trattamento medico.
\subsection{Descrizione delle features raccolte}

La diagnosi dell'HCV può avvenire attraverso diversi metodi e, da qui, la scelta di avere determinati dati riguardanti i pazienti sottoposti ad un trattamento medico. Infatti le infezioni vengono scoperte grazie a indagini riguardanti elevati livelli di enzimi epatici tra i quali l'analina transaminasi (\textbf{ALT}) e l'aspartate transaminase (\textbf{ASP}). L'ALT è stato analizzato per ogni paziente dopo \textit{1, 4, 12, 24, 32 e 64} settimane durante il trattamento per l'HCV per monitorare il paziente stesso. Un altro fattore chiave è stato quello di rilevare l'RNA che contraddistingue il virus, monitorando questo valore fin dall'inizio (\textbf{RNA base}), a 4 e a 12 settimane dall'inizio del trattamento e alla fine (\textbf{RNA EOF}, end-of-treatment). Effettivamente dopo 12 settimane se il valore di RNA del virus dovesse essere molto alto, vorrebbe dire che il paziente è afflitto da una cirrosi cronica. Oltre a questi fattori, sono stati inclusi nel dataset: \textit{età, genere, indice di massa corporea (BMI), febbre, Nausea, mal di testa, diarrea, stanchezza e dolori alle ossa, ittero, piastrine, dolore epigastrero e i livelli di emoglobina e varie cellule (leucociti, ovvero WBC, e globuli rossi, ovvero RBC)}. Nel dataset sono presenti altri valori potenzialmente rilevanti ai fini della classificazione come il \textbf{Baseline histological Grading} ottenuto tramite biopsie epatiche per determinare il grado di danno del fegato. Come menzionato precedentemente, il target value è sicuramente lo stadio di fibrosi che affligge il paziente, ovvero il Baselinehistological staging che è stato classificato attraverso 4 valori, dal meno grave al più grave: \textbf{portal fibrosis, few septa, many septa e cirrhosis}.
\newline Le librerie python utilizzate sono state: NumPy, Pandas, Sklearn, Matplot e SeaBorn.\cite{sklearn}


